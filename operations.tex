
\section{Operations}
\label{sec:operations}
We present some of the algebraic constructions that can be extracted automatically from a theory 
presentation. These are the operations that our library will generate. Our current list is a starting 
list. We believe there are more constructions can be generated and will be adding them to the list 
as we proceed with the work. 

The algorithms to generate these constructions will take as input a flattened theory presentation 
consisting of a set of declarations. Based on our discussion in the previous section, we define a 
theory presentation as a triplet $(S,F,P)$ where $S$ is a set of types, $F$ is a set of function 
symbols and $P$ is a set of propositions.
%\ednote{I believe this is not how we want to define it. The three components need to be there, 
%but we 
%need to somehow reflect the idea that this is a telescope and the order matter. I don't know how 
%to 
%do 
%that.}
We now start defining the operations in terms of how they make use of these three components. 
%Formal definitions are taken from \cite{ehrig2012fundamentals} and \cite{wechler1992universal}
\begin{itemize}
		\item The \textit{signature} of a theory describes the language presented by the theory and 
		consists of sorts and function symbols. A signature is obtained from a theory by 
		dropping the axioms. Therefore, given a theory $(S,F,P)$, the signature of this theory is the 
		tuple $(S,F)$. 
		\item The theory $B$ is a \textit{sub-algebra} of $A$, if operations of $A$ are restricted to a 
		closed subset $B$. Given an algebra $(S,F,P)$, the sub-algebra is defined as 
		$(S^\prime,F^\prime,P^\prime)$, where 
		$\forall s \in S \cdot \exists s^\prime \in S^\prime \cdot s^\prime \subseteq s$ and all 
		functions and axioms in $F^\prime$ and $P^\prime$ are changed accordingly to reflect the 
		new sorts. A sub-algebra is called minimal if it contains no proper sub-algebras. 
		\item The algebra $A \times B = (S,F,P)$ is a \textit{product algebra} of the two algebras $A = 
		(S_A,F_A,P_A)$ and $B=(S_B,F_B,P_B)$ if the sorts $s_i \in S = s_{a_i} \times s_{b_i}$ and all 
		function symbols and axioms are changed accordingly.   
        \item A \textit{projection} from a product algebra to one of its components. 
        \item A \textit{congruence} relation $R$ on an algebra $A$ with respect to a set of operations $S$ 
        is an equivalence relation on $A$ that preserves its structure. A structure-preserving relation 
        $\rightarrow$ has the property that for every operation symbols, $\sigma \in S$, of the algebra, 
        the following axiom holds: \\
        $x_1 \rightarrow y_1, \cdots, x_n \rightarrow y_n$ implies 
        $\sigma(x_1,\cdots x_n) \rightarrow \sigma(y_1,\cdots y_n)$\\
        The relation induced by a homomorphism is called an induced congruence. 
		\item The \textit{quotient algebra} of an algebra $A$ given by a congruence relation $R$ 
		(a.k.a the factorization of $A$ by $R$), 
		is the algebra whose sorts and operations are described with respect to the 
		congruence classes created by the relation $R$. A congruence class $[a]$ is defined $\{b \in 
		s_i | (a,b) \in R\}$
		\item The \textit{record} definition of the theory, in which each declaration may depend only on 
		those before it. 
		%\item Composition operator on instances of theories.
		%\item Signature expansion
		%\item Natural map of a congruence
		
    	\item The \textit{homomorphism} of an algebra $A$ can be obtained by defining two 
    	instances of A,  
    	$x \; y : A$, as well as a function that maps elements of the two instances together $h : x 
    	\rightarrow y$. Then adding the preservation axiom for each function symbol. The 
    	preservation axiom for a function symbol $f_x$ that maps to $f_y$ states that 
    	$h(f_x(a_1, \ldots, a_n))) = f_y(h(a1), \ldots, h(an))$
    	\item The \textit{homomorphism equality} defined as the extensional equality between the two 
    	functions. 
    	\item The \textit{composition} of homomorphisms
    	\item The \textit{kernel} of a homomorphism is a congruence relation $R$ associated with 
    	the 
    	homomorphism $f$ from $A$ to $B$ by the rule $R \; x \; y$ iff  $h(x)=h(y)$
    	\item The \textit{isomorphism} from algebra $A$ to algebra $B$is a homomorphism $h : A 
    	\rightarrow B$ such that a homomorphism $g : B \rightarrow A$ exists and $f \circ g = id_B$ 
    	and $g \circ f = id_A$.
    	\item The \textit{endomorphism} is a homomorphism from algebra $A$ to itself, $h : A 
    	\rightarrow A$ . 
    	\item The \emph{automorphism} is an isomorphism from algebra $A$ to itself, $I : A 
    	\rightarrow A$. 
    	%\note{Y: On P.203, there is a defintion of forgtful functor as the mapping between  
    	%$Cat(SIG^\prime)$ to $Cat(SIG)$. The definition is based on the definition of a category, not an 
    	%algebra. I thought we may consider it for further extensions of the system.}
    	
		\item The \emph{term language} induced by the theory. This language consists of the set of 
		variables, constants and function symbols of the theory. 
		\begin{comment}
		Terms of algebra: The set of terms $T_{OP,s}(X)$ of sort $s \in S$ for the algebra $(S, OP)$ 
		using the set $X$ of variables is defined as
		\begin{itemize}
			\item $X_s \cup K_s \subseteq T_{OP,s}(X)$ where $K_s$ is the set of constant symbols of sort $s$ and $X_s$ is the set of variables of sort $s$. 
			\item $N(t1, \ldots, tn) \in T_{OP,s}(X)$ for all operation symbols $N \in OP$ with $N : s_1 \ldots s_n \rightarrow s$ and all terms $t1 \in T_{OP,s1}(X), \ldots, tn \in T_{OP,sn}(X)$.
			\item There are no further terms of sort $s$ in $T_{OP,s}(X)$. 
		\end{itemize}
		\end{comment}
		\item The \emph{ground terms} of algebra is the term language when the set of variable is 
		empty. 
		\item The \emph{staged terms} of the algebra which are used for partial evaluation in a 
		multi-staged programming setup 
		\item The \emph{structural induction} axiom on the term language to assert a proposition 
		$p(x)$ is true for all terms of the language. For statement $p(x)$ to be true, we need to prove 
		that it is true for constants and variables, then inductively for the term $N(t1, \ldots, tn)$ by 
		proving that $p(t1)$, $p(t2)$, $\ldots$ and $p(tn)$ holds. This is the definition of weak structural 
		induction. We want to generate both weak and strong structural induction. 
		\item The \emph{evaluation} of a term is defined in terms of an \verb|eval : |$T \rightarrow A$ 
		function, where $T$ is the type of terms. If we consider open terms, which contain variables, 
		we need to have a variable assignment function \verb|assign :| $X \rightarrow A$. The 
		function is defined recursively based on the definition of the term language. 
		\begin{comment}
		Let $T_{OP}$ be the set of ground terms of a signature $SIG = (S, OP)$ and $A$ a $SIG-$Algebra. The evaluation function $eval : T_{OP} \rightarrow A$ is recursively defined by 
		\begin{itemize}
			\item $eval(N) = N_{A}$ for all constant symbols $N \in K$. 
			\item $eval(N(t1, \ldots, tn)) = N_{A}(eval(t1), \ldots, eval(tn))$ for all $N(t1, \ldots,tn) \in T_{OP}$. 
		\end{itemize}
		Given an assignment function $ass : X \rightarrow A$ with $ass(x) \in A_{s}$ for $s \in S$, the extension $\overline{ass} : T_{OP}(X) \rightarrow A$ of the assignment $ass : X \rightarrow A$ is recursively defined by 
		\begin{itemize}
			\item $\overline{ass}(x) = ass(x)$ for all variables $x \in X$
			\item $\overline{ass}(N) = N_A$ for all constant symbols $N \in K$. 
			\item $\overline{ass}(N(t1, \ldots, tn)) = N_A(\overline{ass}(t1), \ldots, \overline{ass}(t2))$ for all function symbols in $T_{OP}(X)$. 
		\end{itemize}
		\end{comment} 
		\item A \emph{simplify} function that applies axioms as term rewrite rules. For example, the 
		unit axiom can result in reducing the term by removing the unit part of it. The function applies 
		simplification rules until no more rules can be applied. 
		\begin{comment}
		\item Equational Specification \\
		A specification $SPEC = (S, OP, E)$ consists of a signature $SIG = (S, OP) $ and a set $E$ of equations $e$ w.r.t $SIG$. 
		An algebra $A$ over the specification $SPEC$ is a $SIG$- algebra $A$ which satisfies all 
		equations in $E$. 
		\end{comment}
		\item The \emph{rewrite rules} defined on equations based on axioms. 
	    Given a set $E$ of equations with a fixed set of variables $X = X_e$, each equation $e = (L,R) 
	    \in E$ defines two substitution rules $L \Rightarrow R$ and $R \Rightarrow L$. \newline
		The application of a substitution rule $t1 \Rightarrow t2$ to a term $t$ is valid if there is an 
		extension $\cn{assign} : T_{OP}(X) \rightarrow T_{OP}(X)$ s.t. 
		\begin{itemize}
			\item $\overline{t1} = \cn{assign}(t1)$.
			\item $\overline{t2} = \cn{assign}(t2)$.
			\item $\overline{t1}$ is a sub-term of $t$. 
		\end{itemize}
		Therefore, the substitution yields a term $t^{\prime}$ defined as $t^{\prime} = t(\overline{t1}/\overline{t2})$\newline
		$t \Rightarrow_{E}^{*} t^{\prime}$ is called a derivation from $t$ to $t^{\prime}$ via $E$. 
		\item The set of \emph{sub-terms} of a term in the language, where a 
		term $t1$ is a sub-term of $t2$ if $t1 \leq t2$. The $\leq$ relation is defined as: 
		\begin{itemize}
			\item $t \leq t$ for all $t \in T_{OP}(X)$
			\item $t \leq N(t1, \ldots, tn)$ if $t = t_i$ for some $i \in {1, \ldots, n}$ and $N(t1, \ldots tn) \in T_{OP}(X)$
			\item If $t1 \leq t2$ and $t2 \leq t3$ then $t1 \leq t3$. 
		\end{itemize}
		\item The \textit{equivalence} of two terms, where $t1 = t2$ iff $\cn{eval}_A(t1) = 
		\cn{eval}_A(t2)$ 
		for all algebras $A$
		\begin{comment}
		\item The \textit{quotient} term algebra \ednote{Is this different than the quotient algebra?}
	Given a specification $SPEC$, the quotient term algebra $T_{SPEC} = ((Q_{s})_{s\in S}, 
	(N_Q)_{N \in OP})$ can be defined as
	\begin{itemize}
	\item For each sort $s \in S$, there is a base set $Q_s = \{[t]$  $| t \in T_{OP,s}\}$ where 
	the congruence class of $t$ s defined as $[t] = \{t^{\prime}$  $|  t^{\prime} \equiv t\}$
	\item For each constant symbol $N : \rightarrow s$ in $OP$, the constant $N_Q$ is the 
	congruence class generated by $N$; $N_Q = [N]$
	\item For each operation symbol, $N : s1, \ldots, sn \rightarrow s$, in $OP$, the operation 
	$N_Q : Q_{s1} \times \ldots \times Q_{sn} \rightarrow Q_{s}$ is defined as $N_Q([t1] \times 
	\ldots \times [tn]) = [N(t1, \ldots tn)]$. 
	\end{itemize}	
			\end{comment}
		\item The \emph{parse tree} for a term, where the term is represented as an instance of the 
		parse tree datatype 
		\item The homomorphism between terms and trees.
%		\item Evaluation mappings \ednote{check the definition}
\end{itemize}